
\documentclass[12pt, twocolumn]{article}
% font size could be 10pt (default), 11pt or 12 pt
% paper size coulde be letterpaper (default), legalpaper, executivepaper,
% a4paper, a5paper or b5paper
% side coulde be oneside (default) or twoside 
% columns coulde be onecolumn (default) or twocolumn
% graphics coulde be final (default) or draft 
%
% titlepage coulde be notitlepage (default) or titlepage which 
% makes an extra page for title 
% 
% paper alignment coulde be portrait (default) or landscape 
%
% equations coulde be 
%   default number of the equation on the rigth and equation centered 
%   leqno number on the left and equation centered 
%   fleqn number on the rigth and  equation on the left side
%	
\title{Reclaiming a Waveguide Switch -- An Adventure In 3D Printing}
\author{Matt Reilly  \\
	kb1vc \\
	}

\date{\today} 
% \date{\today} date coulde be today 
% \date{25.12.00} or be a certain date
% \date{ } or there is no date 
\begin{document}
% Hint: \title{what ever}, \author{who care} and \date{when ever} could stand 
% before or after the \begin{document} command 
% BUT the \maketitle command MUST come AFTER the \begin{document} command! 
\maketitle


\begin{abstract}
  The story you are about to read is true.  The names have been
  changed to protect the innocent.
\end{abstract}

%\tableofcontents % create a table of contens 



\section{Introduction}
So began each episode of ``Dragnet'' both on radio and television.
This is the story of a project to reclaim a broken WR90 waveguide
switch.  No names have been changed, however, because nobody was innocent.

\section{Starting Simple}
Before we dive in to the construction of the waveguide switch,
it may help to look at a much simpler 3D printing project.

Quite a while back, I bought an HP6289A DC power supply at a hamfest.
It was a real find, but at some point in its checkered past it had
been dropped on its face.
The fall shattered the two fine adjustment knobs.  These were
concentric with the coarse adjustment knobs on two shafts: one for
voltage and the second for current.
For years, the absence of the knobs was no great annoyance. But one night
I decided that making replacements might be a short and simple project.

This kind of project really points out the virtues of an inexpensive
3D printer.  No great precision is needed, and the cost of a mistake
is a small amount of material and about 15 minutes of waiting for the
printer to finish.
In fact, it took two tries to get the design right. (Though had I
been paying attention, it could have been right the first time --
measure twice, print once.) But having the printer in the shack
means that I got through two iterations in less than an hour.

In the ``old days'' I'd probably have done a drawing like the one
shown in Figure \ref{fig_old_drawing}.

\begin{figure}

\caption{Traditional Mechanical Drawing for HP6289 Vernier Knob}
\end{figure}

\begin{thebibliography}{9}
\bibitem[Doe]{doe} \emph{First and last \LaTeX{} example.},
John Doe 50 B.C. 
\end{thebibliography}

\end{document}

